% Chapter 2: Literature Review

\section{Introduction}

The field of tourist itinerary planning intersects multiple research domains including operations research, artificial intelligence, human-computer interaction, and tourism studies. This chapter systematically reviews the evolution of approaches from early mathematical formulations to modern AI-driven systems, identifying the gaps that motivate our quality-based ranking framework.

\section{Foundational Work in Itinerary Planning}

\subsection{The Orienteering Problem}

The mathematical foundation for tourist trip planning originates from the Orienteering Problem (OP), introduced by Tsiligirides (1984). The OP seeks to find a path through a subset of locations that maximizes collected scores while respecting a time budget. Vansteenwegen et al. (2011) provide a comprehensive survey showing that even simplified versions are NP-hard.

Key OP variants relevant to tourism include:
\begin{itemize}
    \item Team Orienteering Problem (TOP): Multiple tourists with shared objectives
    \item Time-Dependent OP (TDOP): Varying travel times throughout the day
    \item OP with Time Windows (OPTW): POI opening hours constraints
    \item Stochastic OP (SOP): Uncertain travel times and POI availability
\end{itemize}

\subsection{Interactive Planning Systems}

Basu Roy et al. (2011) introduced the seminal framework for interactive itinerary planning, establishing the three-phase cycle:
\begin{enumerate}
    \item Algorithm generates k candidate POIs
    \item User provides feedback (accept/reject)
    \item System updates preferences and repeats
\end{enumerate}

This work proved NP-completeness even for simple scoring functions and developed the GreedyPOISelection and HeapPrunGreedyPOI algorithms that form baselines for our research.

\section{User Preferences and Personalization}

\subsection{Preference Learning}

Lim et al. (2018) analyzed tourist behavior across 8 cities, revealing that users typically prefer 3-7 POIs per day with significant variance based on travel style. Their work established preference categories (cultural, adventure, leisure) that inform our user profiling.

Key findings on tourist preferences:
\begin{itemize}
    \item 67\% of tourists prioritize quality over quantity
    \item Visit duration varies 40-180 minutes based on POI type
    \item Social influence affects 45\% of POI choices
    \item Weather sensitivity impacts 72\% of outdoor attractions
\end{itemize}

\subsection{Context-Aware Recommendations}

Yuan et al. (2013) pioneered time-aware POI recommendations, showing that temporal patterns significantly affect tourist satisfaction. Morning preferences differ from evening activities, and weekday patterns vary from weekends.

\section{Algorithmic Approaches}

\subsection{Exact Algorithms}

While exact solutions using Integer Linear Programming (ILP) guarantee optimality, they become intractable beyond 20-30 POIs. Gavalas et al. (2014) surveyed exact methods showing exponential scaling even with sophisticated branch-and-bound techniques.

\subsection{Heuristic Methods}

Practical systems rely on heuristics for scalability:
\begin{itemize}
    \item \textbf{Greedy Algorithms:} Fast but potentially suboptimal
    \item \textbf{Local Search:} Iterative improvement with local optima risks
    \item \textbf{Metaheuristics:} Genetic algorithms, simulated annealing, ant colony optimization
    \item \textbf{Hybrid Approaches:} Combining multiple strategies
\end{itemize}

Souffriau et al. (2010) demonstrated that path relinking achieves near-optimal solutions 87\% faster than exact methods for tourist-scale problems.

\section{Quality Metrics and Evaluation}

\subsection{Traditional Metrics}

Early work focused on simple metrics:
\begin{itemize}
    \item Coverage: Number of POIs visited
    \item Distance: Total travel distance/time
    \item Score: Sum of POI ratings or utilities
\end{itemize}

These fail to capture tourist satisfaction complexity, leading to technically optimal but practically poor itineraries.

\subsection{Composite Quality Measures}

Recent research advocates multi-criteria evaluation. Our CSS metric builds on:
\begin{itemize}
    \item Satisfaction scores from Cao et al. (2020)
    \item Time efficiency from Zheng et al. (2020)
    \item Diversity metrics from Friedman \& Dieng (2023)
    \item Feasibility constraints from Chen et al. (2021)
\end{itemize}

\section{Dynamic and Real-Time Planning}

\subsection{Incremental Algorithms}

Koenig et al. (2004) introduced LPA* for efficient replanning, reusing previous search efforts when environments change. This achieves orders-of-magnitude speedups for dynamic scenarios.

\subsection{Real-World Dynamics}

Huang et al. (2024) studied real-world disruptions affecting tourist itineraries:
\begin{itemize}
    \item 23\% experience weather-related changes
    \item 18\% face unexpected POI closures
    \item 31\% modify plans due to crowd levels
    \item 42\% adjust based on real-time discoveries
\end{itemize}

\section{Recent AI/ML Approaches}

\subsection{Deep Learning for Tourism}

Sun et al. (2020) applied LSTMs to model sequential POI preferences, achieving 34\% improvement in next-POI prediction. However, black-box nature limits explainability crucial for tourist trust.

\subsection{Large Language Models}

TravelPlanner (Xie et al., 2024) benchmarked LLM-based agents on complex itinerary planning, revealing a sobering 0.6\% success rate on constraint-heavy queries. This highlights the gap between impressive language capabilities and structured planning requirements.

\section{Research Gaps}

Our literature review identifies four critical gaps:

\textbf{1. Quality vs. Coverage Trade-off:} Existing work optimizes for visiting more POIs rather than meaningful experiences.

\textbf{2. Static Planning Assumptions:} Most algorithms assume fixed conditions, ignoring real-world dynamics.

\textbf{3. Monolithic Metrics:} Single-objective optimization fails to capture multi-faceted tourist satisfaction.

\textbf{4. Scalability Barriers:} Optimal algorithms don't scale; scalable algorithms aren't optimal enough.

\section{Summary}

The evolution from operations research to AI-driven approaches has advanced tourist itinerary planning significantly. However, achieving human-competitive performance requires bridging mathematical optimization with user experience design. Our work addresses this by introducing quality-based ranking that balances algorithmic efficiency with tourist satisfaction, setting new standards for practical deployment.

The next chapter presents our methodology for achieving these goals through the CSS metric and hybrid algorithmic framework.