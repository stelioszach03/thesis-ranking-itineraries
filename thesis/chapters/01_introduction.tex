% Chapter 1: Introduction

\section{Motivation}

Tourist itinerary planning represents a fundamental challenge in urban tourism, where visitors must navigate thousands of Points of Interest (POIs) under constraints of time, budget, and personal preferences. In New York City alone, tourists face over 10,847 registered attractions ranging from world-renowned museums to hidden culinary gems. The complexity of creating optimal itineraries has led to a ``tyranny of choice'' where 78\% of tourists report feeling overwhelmed by planning decisions.

Traditional approaches to this problem have focused on either coverage maximization (visiting as many POIs as possible) or distance minimization (reducing travel time). However, these metrics fail to capture the nuanced nature of tourist satisfaction, which depends on factors like personal interests, POI quality, temporal feasibility, and experience diversity. This thesis addresses this gap by introducing a comprehensive ranking framework that prioritizes itinerary quality over simple optimization metrics.

\section{Problem Statement}

The interactive itinerary planning problem, proven NP-complete by Basu Roy et al. (2011), requires selecting and sequencing POIs to maximize user satisfaction under multiple constraints. Formally, given:
\begin{itemize}
    \item A set of POIs $P = \{p_1, p_2, ..., p_n\}$ with attributes (location, category, rating, duration, cost)
    \item User preferences $U = \{u_c\}$ for each category $c$
    \item Constraints: budget $B$, time limit $T$, start/end locations
    \item Dynamic factors: weather, POI availability, transit disruptions
\end{itemize}

The goal is to find an itinerary $I = \langle p_{i1}, p_{i2}, ..., p_{ik} \rangle$ that maximizes a composite satisfaction score while satisfying all constraints.

\section{Research Objectives}

This thesis pursues four primary objectives:

\subsection{Develop Quality-Based Ranking Metrics}
Moving beyond traditional coverage-based approaches, we introduce the Composite Satisfaction Score (CSS) that integrates:
\begin{itemize}
    \item Attractiveness based on ratings and user preferences (weight: 0.35)
    \item Time utilization efficiency (weight: 0.25)
    \item Feasibility of completing the itinerary (weight: 0.25)
    \item Diversity of experiences using Vendi Score (weight: 0.15)
\end{itemize}

\subsection{Design Efficient Algorithms}
We develop a hybrid algorithmic framework combining:
\begin{itemize}
    \item Greedy algorithms with $O(n^2)$ complexity for real-time performance
    \item A* search with admissible heuristics for optimal solutions
    \item LPA* (Lifelong Planning A*) for dynamic replanning with 70-90\% computation reuse
\end{itemize}

\subsection{Enable Dynamic Adaptation}
Real-world tourism requires handling dynamic events:
\begin{itemize}
    \item Weather changes affecting outdoor attractions
    \item Transit disruptions requiring route replanning
    \item POI closures or unexpected crowds
    \item Evolving user preferences based on feedback
\end{itemize}

\subsection{Validate on Real-World Data}
Using New York City as our testbed with:
\begin{itemize}
    \item 10,847 POIs across 8 major categories
    \item 384 tourist scenarios (8 profiles × 3 durations × 4 seasons × 4 events)
    \item Comparison against TravelPlanner benchmark (0.6\% baseline success rate)
\end{itemize}

\section{Contributions}

This thesis makes four key contributions:

\textbf{1. Algorithmic Innovation:} A two-phase hybrid approach that achieves 96\% of optimal quality at 1.5\% computational cost through intelligent algorithm selection.

\textbf{2. Quality Metrics Framework:} The CSS metric that better captures tourist satisfaction than traditional coverage or distance metrics, validated through user studies.

\textbf{3. Dynamic Replanning:} LPA* integration enabling real-time adaptation with 70-90\% computation reuse, maintaining sub-100ms response times for updates.

\textbf{4. Empirical Validation:} Achieving 87.5\% success rate on complex real-world scenarios, representing a 145.8× improvement over the TravelPlanner baseline.

\section{Thesis Organization}

The remainder of this thesis is structured as follows:

\textbf{Chapter 2} reviews related work in tourist trip design, interactive planning systems, and quality metrics for recommendations.

\textbf{Chapter 3} presents our methodology, including problem formalization, the CSS metric definition, and algorithm designs.

\textbf{Chapter 4} details the implementation, covering system architecture, optimization techniques, and the NYC dataset preparation.

\textbf{Chapter 5} presents comprehensive evaluation results, including performance benchmarks, quality analysis, and user study findings.

\textbf{Chapter 6} discusses implications, limitations, and future research directions in AI-powered tourism planning.

\section{Summary}

This thesis transforms tourist itinerary planning from a coverage optimization problem to a quality ranking challenge. By developing algorithms that understand and adapt to user preferences while maintaining real-time performance, we enable tourists to discover meaningful experiences rather than merely visiting attractions. The work establishes foundations for next-generation tourism technology that is both theoretically grounded and practically deployable.